\subsection{Clone Repository}

The first step to getting starting with this project is to clone the 
repository from its 'git'.

Ensure that you have NodeJS available on your system. You will
need it to host both the Front-End and the Back-End of the application.

Ensure your mail server is active, this application sends confirmation
emails to:
\begin{itemize}
    \item activate accounts
    \item reset passwords
    \item update details
    \item notify of comments \& likes
    \item notify on a Match
\end{itemize}

\subsection{Backend}

The backend can be located by going to the 'api' folder inside the
repository you cloned. You will need MySQL to be installed
independently so as to run a Relational Database.

After changing directory until you are in the same folder as 'package.json'. You will run'npm install',
this will allow the npm to install all dependancies required to run the applicaton.

\textbf{NB:} Ensure that port 4500 is free and that no service is currently using it, a new port
can be defined in 'api/server.js'

Run 'node start.js' at least once. This will create the Database environment
and begin seeding it with users.

Run 'npm start' and if this has run successfully, the console
will print:
    >Matcha has started running/listening on port:4500!

\subsection{Frontend}

This section is about fine-tuning the front-end and how to set it up and get it running.
First ensure that you have NodeJs and NPM installed.

Navigate to the UI folder and run 'npm install'.

Inside the same directory you can run:
'npm start'. This runs the application in development mode.

Open 'http://localhost:3000' to view the running application in the browser.

The page will reload if you make edits to the client side code. It is recommended not to
run it in development mode when the business is using it. This may cause harm to clients.

We recommended run the application in production mode.

\subsubsection*{Production Mode}

In the UI folder, ensure port 5000 is open. This is vital in order to build the project.
The project can be built for production mode. This correctly bundles React and optimises for
the best performance.

The build is minified and the filenmaes include the hashes.

Your app is ready to be deployed!